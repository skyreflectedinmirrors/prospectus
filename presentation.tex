\documentclass{beamer}
\usepackage[citestyle=authoryear, style=authoryear, backend=biber, uniquename=init, giveninits=true]{biblatex}
\usepackage{lmodern}
\usepackage{amsmath,amssymb}
\usepackage{mathtools}
\usepackage[version=3]{mhchem} % Formula subscripts using \ce{}, e.g., \ce{H2SO4}
\usepackage{siunitx}
\usepackage[utf8]{inputenc}
% https://tex.stackexchange.com/a/351011/56227
\usepackage{threeparttable}
\usepackage{caption}
\captionsetup[figure]{labelformat=empty,justification=centering}
% https://tex.stackexchange.com/a/155611/56227
\usepackage{textcomp}
% footnote per page https://tex.stackexchange.com/a/1088/56227
\usepackage{perpage}
% speaker notes
\usepackage{pgfpages}
\usepackage{hyperref}

\setbeameroption{hide notes} % only slides
% \setbeameroption{show only notes} % only notes
% \setbeameroption{show notes on second screen=right} % both

\graphicspath{{./Figures/}}

% clear title and doi from citations
% from https://tex.stackexchange.com/q/165481/56227
\AtEveryCitekey{\clearfield{doi}\clearfield{title}}

% use dash for range phrase by default, and one unit
\sisetup{range-phrase=--, range-units = single}

% turn off nav bar
\beamertemplatenavigationsymbolsempty

% footnote per page https://tex.stackexchange.com/a/1088/56227
\MakePerPage{footnote}

% symbols for footnotes
\renewcommand{\thefootnote}{\fnsymbol{footnote}}

%smaller footnotes, adapted from http://tex.stackexchange.com/a/146021
\setbeamerfont{footnote}{size=\fontsize{6pt}{0pt}}
\setbeamerfont{mpfootnote}{size=\fontsize{6pt}{0pt}}

% numeric footnote symbols
\def\thefootnote{\arabic{footnote}}
\def\thempfootnote{\arabic{mpfootnote}}

% set themes
\usetheme{Frankfurt}
\usecolortheme{seahorse}
\usefonttheme{serif}

% adjust footnote width
%\newlength{\tmpwidth}
%\setlength{\tmpwidth}{0.65\paperwidth}
%\addtobeamertemplate{footnote}{\hsize\tmpwidth}{}

%footnote spacing avoids navigation / margins, adapted from http://tex.stackexchange.com/a/44231
%\addtobeamertemplate{footnote}{\vspace{-6pt}\advance\hsize-0.5cm}{\vspace{6pt}}
%\makeatletter
% Alternative A: footnote rule
%\renewcommand*{\footnoterule}{\kern -3pt \hrule \@width 2in \kern 8.6pt}
% Alternative B: no footnote rule
% \renewcommand*{\footnoterule}{\kern 6pt}
%\makeatother

%make frame for each section
\AtBeginSection[]{
  \begin{frame}
  \vfill
  \centering
  \begin{beamercolorbox}[sep=8pt,center,shadow=true,rounded=true]{title}
    \usebeamerfont{title}\insertsectionhead\par%
  \end{beamercolorbox}
  \vfill
  \end{frame}
}

%auto incrementing titles from http://tex.stackexchange.com/a/231533
\newcounter{expensive}
\newcommand\ExpTitle{%
  \frametitle{\refstepcounter{expensive}{Chemical kinetic integration is \textbf{expensive}~--}~\theexpensive}}
\resetcounteronoverlays{expensive}

% define background template env
\newenvironment{background}{%
\usebackgroundtemplate{%
\rule{0pt}{\paperheight}%
\hspace*{\paperwidth}%
\makebox[0pt][r]{\includegraphics[width=35mm]{logo}}%
}}{}

% space between paragraphs beamer
% https://tex.stackexchange.com/a/22644/56227
\def\parend/{\\~\\}

\bibliography{presentation.bib}

%opening
\title{Accelerating reacting flow simulations via vectorized chemical kinetic integration}
\author{Nick Curtis}
\institute{University of Connecticut}
\date{\today}

\begin{document}

\begin{background}
\maketitle
\end{background}

\begin{background}
\section{Introduction}
\end{background}

\begin{frame}
\frametitle{Improving combustion technology}
The push to design new clean, efficient combustion technologies has driven both:
\begin{itemize}
 \item the development of combustion devices operating in new regimes such as low-temperature combustion as well as,
 \item the creation of large detailed chemical kinetic models for a wide range of fuel classes from bioalcohols~\footfullcite{SARATHY2009852} to gasoline surrogates\footfullcite{SARATHY201867}.
\end{itemize}
These next generation combustion devices are often controlled by chemical processes, rather than current methods which rely on directly physical processes.
\begin{itemize}
 \item Fuel selection and composition are key parameters for the design of such technologies, and the use of realistic chemical modeling is critical in order to achieve predictive numerical simulations.
\end{itemize}
\end{frame}

\begin{frame}
 \frametitle{Challenges of reactive-flow simulations}
 \textbf{Model size:} Chemical kinetic models for transportation and energy relevant fuels may consist of hundreds to thousands of chemical species, with potentially tens of thousands of reactions (e.g., gasoline \footfullcite{Mehl:2011cn} and jet fuel\footfullcite{Naik2011434})\parend/
 \textbf{Range of scales:} A typical reactive-flow simulation encompasses a wide range of physical and temporal scales, e.g., the flow-through time of a combustion chamber is typically on the order of seconds to milliseconds, while chemical timescales may be just picoseconds in duration\footfullcite{Lu:2009gh} due to the presence of highly reactive radicals and associated short chemical timescales.\parend/
\end{frame}

\begin{frame}
 \frametitle{Numerical stiffness}
 The large range of time-scales present in these chemical kinetic models---known as numerical stiffness---has traditionally been handled by use of implicit integration techniques, which require repeated evaluation and factorization of the chemical kinetic Jacobian.
 \begin{itemize}
  \item Naive implementations of these operations scale \textbf{quadratically} and \textbf{cubically} with the number of species in a model, respectively\footfullcite{Lu:2009gh}.
 \end{itemize}
 However, using even modestly sized chemical kinetic models with such a solver can incur severe computation cost (e.g.,~\footfullcite{Moiz2016123}) for realistic reactive-flow simulations.
\end{frame}


\begin{frame}
 \frametitle{Reducing the cost of chemical kinetics}
  A host of techniques have been developed to reduce the cost of chemical kinetic calculations while maintaining fidelity~\footfullcite{turanyi2016analysis}, e.g.:
  \begin{itemize}
   \item removal of unimportant species and reactions,
   \item lumping of species with similar thermochemical properties, and
   \item time-scale methods that reduce numerical stiffness
  \end{itemize}
  Today we will focus on two such (complementary) techniques, analytical Jacobian evaluation and vectorized chemical kinetic integration.
\end{frame}

\begin{background}
\section{Analytical jacobian evaluation}
\end{background}

\begin{frame}
\frametitle{Why use an analytical jacobian?}
\begin{itemize}
 \item Exact chemical kinetic Jacobian evaluation is required for accuracy in both computational diagnostic methods (e.g., CSP, or CEMA) as well as non Newton--Krylov based implicit integration techniques\footfullcite{HANSEN2018257} (e.g., a linearly-implicit Rosenbrock methods).
 \item In addition, analytical Jacobian evaluation drops the cost of Jacobian evaluation from a quadratic dependence on the number of species to a linear dependence on the number of reactions in a chemical model\footfullcite{Lu:2009gh}.
 \item Further, carefully chosen system of equations can greatly increase the sparsity of the Jacobian~\footfullcite{Schwer2002270}, allowing for fast, sparse linear-algebra and matrix factorization techniques to be used.
\end{itemize}
\end{frame}

\begin{frame}
 \frametitle{\texttt{pyJac} - an analytical jacobian generator}
 \begin{columns}[c]
  \begin{column}{0.6\textwidth}
  \begin{minipage}[c]{\columnwidth}
    \texttt{pyJac}\footfullcite{Niemeyer:2016aa} is an open-source, validated, chemical kinetic source-rate and analytical Jacobian code-generation platform with thousands of downloads, and has been used in applications such as:
    \begin{itemize}
     \item LES studies of duel-fuel spray combustion (right),
     \item validation\slash performance testing for a DNS code for heterogeneous processors~\footfullcite{HERNANDEZPEREZ201873},
     \item derivation of non-ambiguous derivatives in selection of chemical kinetic state vector~\footfullcite{HANSEN2018257}
    \end{itemize}
    \vfill
  \end{minipage}
  \end{column}
  \begin{column}{0.4\textwidth}
    \begin{center}
     \begin{figure}
      \centering
      \includegraphics[width=\columnwidth]{spray.png}
      \caption{OpenFOAM LES ECN-Spray A simulation using \texttt{pyJac}}
     \end{figure}
    \end{center}
  \end{column}
 \end{columns}
 \note[item]{The Spray-A simulation is from an upcoming manuscript from Aalto University}
\end{frame}

\begin{frame}
 \frametitle{\texttt{pyJac-V2} - a \textit{better} analytical jacobian generator}
 The initial version of \texttt{pyJac} was capable of vectorized-GPU and multithreaded-CPU operation, \textit{however}:
 \begin{itemize}
  \item no vectorized execution on the CPU (or other accelerators)
  \item the selection of state variables in \texttt{pyJac-V1} meant that the resulting jacobian was almost completely dense
 \end{itemize}
 \textrightarrow Enter \texttt{pyJac-V2}
\end{frame}

\begin{frame}
\frametitle{Increased jacobian sparsity}
A change of variables lead to a significant increase in Jacobian sparsity (with options available to further increase sparsity at the expense of strict correctness):
\begin{figure}
\begin{measuredfigure}
 \includegraphics[height=0.5\textheight]{ch4_sparsity_exact.png}
 \caption{Representation of sparsity of GRI-Mech 3.0 chemical kinetic model, a black square indicates a non-zero jacobian entry.}
 
\end{measuredfigure}
\end{figure}
\end{frame}

\begin{frame}
\frametitle{Vectorized evaluation on the CPU}
\begin{columns}
 \begin{column}{0.5\textwidth}
  The new version of \texttt{pyJac} utilizes the OpenCL programming language to achieve vectorization on the CPU, GPU and other accelerators.
  \begin{itemize}
   \item achievable CPU-speedup depends on the \textit{vector-width} of the device in question ($\SI{4}{\times}$ here, up-to $\SI{8}{\times}$ on newest processors).
  \end{itemize}
  Vectorized CPU code was $\SIrange{3.03}{4.23}{\times}$ and $\SIrange{6.63}{9.44}{\times}$ faster than purely parallel code for dense and sparse jacobian evaluation, respectively.
 \end{column}
 \begin{column}{0.5\textwidth}
  \begin{figure}
  \includegraphics[width=\textwidth]{sparse_vs_dense_speedup.pdf}
  \caption{Speedup achieved by CPU-vectorization of \texttt{pyJac-V2}'s sparse and dense jacobian for several chemical models.}
  \end{figure}
 \end{column}
\end{columns}
\end{frame}


\begin{background}
 \section{Vectorized ODE integration}
\end{background}

\begin{frame}
 \frametitle{Operator splitting and vectorization}
 Most reacting-flow codes utilize the operator-splitting techniques (e.g.~\footfullcite{Knio:1999}$^{,}$~\footfullcite{Ren:2008}), separating a large system of coupled partial differential equations (PDEs), such that different physical processes are solved independently.
 \begin{itemize}
  \item Independent initial value problem (IVP) of chemical kinetic ordinary differential equations (ODEs) at each computational location (e.g. cell) in the domain~$\rightarrow$~vectorization!
 \end{itemize}
\end{frame}



\end{document}