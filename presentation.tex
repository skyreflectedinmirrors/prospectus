\documentclass{beamer}
\usepackage[citestyle=authoryear, style=authoryear, backend=biber, uniquename=init, giveninits=true]{biblatex}
\usepackage{lmodern}
\usepackage{amsmath}
\usepackage{mathtools}
\usepackage[version=3]{mhchem} % Formula subscripts using \ce{}, e.g., \ce{H2SO4}
\usepackage{siunitx}
\usepackage[utf8]{inputenc}
% https://tex.stackexchange.com/a/155611/56227
\usepackage{textcomp}
% footnote per page https://tex.stackexchange.com/a/1088/56227
\usepackage{perpage}
\usepackage{hyperref}

\usebackgroundtemplate{%
  \rule{0pt}{\paperheight}%
  \hspace*{\paperwidth}%
  \makebox[0pt][r]{\includegraphics[width=35mm]{logo}}
}%

\graphicspath{{./Figures/}}

% clear title and doi from citations
% from https://tex.stackexchange.com/q/165481/56227
\AtEveryCitekey{\clearfield{doi}\clearfield{title}}

% use dash for range phrase by default, and one unit
\sisetup{range-phrase=--, range-units = single}

% turn off nav bar
\beamertemplatenavigationsymbolsempty

% footnote per page https://tex.stackexchange.com/a/1088/56227
\MakePerPage{footnote}

% symbols for footnotes
\renewcommand{\thefootnote}{\fnsymbol{footnote}}

%smaller footnotes, adapted from http://tex.stackexchange.com/a/146021
\setbeamerfont{footnote}{size=\fontsize{7pt}{0pt}}

% adjust footnote width
\newlength{\tmpwidth}
\setlength{\tmpwidth}{0.65\paperwidth}
\addtobeamertemplate{footnote}{\hsize\tmpwidth}{}

%footnote spacing avoids navigation / margins, adapted from http://tex.stackexchange.com/a/44231
\addtobeamertemplate{footnote}{\vspace{-6pt}\advance\hsize-0.5cm}{\vspace{6pt}}
\makeatletter
% Alternative A: footnote rule
\renewcommand*{\footnoterule}{\kern -3pt \hrule \@width 2in \kern 8.6pt}
% Alternative B: no footnote rule
% \renewcommand*{\footnoterule}{\kern 6pt}
\makeatother

%make frame for each section
\AtBeginSection[]{
  \begin{frame}
  \vfill
  \centering
  \begin{beamercolorbox}[sep=8pt,center,shadow=true,rounded=true]{title}
    \usebeamerfont{title}\insertsectionhead\par%
  \end{beamercolorbox}
  \vfill
  \end{frame}
}

\newcounter{dummynote1}% Save footnote counter
\newcounter{dummynote2}% Save footnote counter

%auto incrementing titles from http://tex.stackexchange.com/a/231533
\newcounter{expensive}
\newcommand\ExpTitle{%
  \frametitle{\refstepcounter{expensive}{Chemical kinetic integration is \textbf{expensive}~--}~\theexpensive}}
\resetcounteronoverlays{expensive}

\newcounter{rosen}
\newcommand\RosenTitle{%
  \frametitle{\refstepcounter{rosen}{Rosenbrock Methods~--}~\therosen}}
\resetcounteronoverlays{rosen}

\newcounter{wmeth}
\newcommand\WTitle{%
  \frametitle{\refstepcounter{wmeth}{W-Methods~--}~\thewmeth}}
\resetcounteronoverlays{wmeth}

\newcounter{twostep}
\newcommand\TwoStepTitle{%
  \frametitle{\refstepcounter{twostep}{Two-Step Rosenbrock-type Methods~--}~\thetwostep}}
\resetcounteronoverlays{twostep}

\newcounter{stiff}
\newcommand\BalanceTitle{%
  \frametitle{\refstepcounter{stiff}{Stiffness-based Load Balancing~--}~\thestiff}}
\resetcounteronoverlays{stiff}

\bibliography{presentation.bib}

%opening
\title{Accelerating chemical kinetic integration for reacting-flows on hybrid architectures: A Literature Review}
\author{Nick Curtis}
\institute{University of Connecticut}
\date{\today}

\begin{document}

\maketitle

\section{Introduction}

\begin{frame}
\frametitle{Realistic chemical modeling is \textbf{critical}}
In order to meet increasingly stringent emissions and efficiency requirements, designers of combustion devices have turned to \textbf{new technologies} and \textbf{new fuels}
\begin{itemize}
 \item Novel combustion regimes such as low-temperature combustion are often controlled by chemical processes, rather than directly controllable physical processes as in current technology
 \item Further, developed solutions must be flexible to accommodate a variety of next generation fuels
\end{itemize}
Computationally guided combustion design has played an important role in development of these new technologies, however use of realistic chemical modeling (required for predictive reacting-flow simulations) is prohibitively expensive for most practical systems. 
\end{frame}

\begin{frame}
 \frametitle{Chemical kinetic integration is \textbf{expensive}}
 Chemical kinetic models for fuels relevant to transportation and energy generation may consist of hundreds to thousands of chemical species, with potentially tens of thousands of reactions.
 \begin{itemize}
  \item e.g., for gasoline \footfullcite{Mehl:2011cn} and jet fuel\footfullcite{Naik2011434}
 \end{itemize}
 Further, chemical kinetic models are typically \textbf{stiff}---due to the presence of highly reactive radicals and associated short chemical timescales.
 \begin{itemize}
  \item Implicit integration techniques are commonly used to efficiently deal with stiffness, requiring repeated evaluation and factorization of the chemical kinetic Jacobian.
  \item Naive implementations of these operations scale \textbf{quadratically} and \textbf{cubically} with the number of species in a model, respectively\footfullcite{Lu:2009gh}.
 \end{itemize}
\end{frame}

%\begin{frame}
% \ExpTitle
% As a result, chemical kinetic integration may be the most computationally intensive portion of reacting flow simulations, e.g. consuming \SI{70}{\percent}~\footfullcite{kodavasal2016development} of the total solver time in each time-step.
%\end{frame}


\begin{frame}
 \frametitle{Strategies for cost reduction}
 A few of the major strategies to accelerate chemical kinetic integration:
 \begin{itemize}
  \item \textbf{Model reduction}, a host of techniques to reduce the size of the system being solved while maintaining accuracy (as compared to the full model)
  \item \textbf{Improved integration techniques}, development of new integration algorithms specifically for chemical kinetics, e.g. hybrid implicit\slash explicit integrators, tabulation techniques, analytical Jacobian codes, and on-the-fly stiffness removal
  \item \textbf{Solver vectorization}, to better utilize available hardware, e.g. via single-instruction, multiple-data\slash multiple-thread (SIMD\slash SIMT) execution on central processing units (CPUs) and other accelerators such as Intel's many integrated core architecture (MIC) or graphics processing units (GPUs).
  \item \textbf{High performance computing techniques}, stiffness-based load balancing, scaling for high performance clusters
 \end{itemize}
 We will focus on SIMD \textbf{vectorization}, \textbf{improved integration} and \textbf{high performance computing} techniques.
\end{frame}

\begin{frame}
 \frametitle{Operator splitting and vectorization}
 Most reacting-flow codes utilize the operator-splitting techniques (e.g.~\footfullcite{Knio:1999}$^{,}$~\footfullcite{Ren:2008}), separating a large system of coupled partial differential equations (PDEs), such that different physical processes are solved independently.
 \begin{itemize}
  \item Independent initial value problem (IVP) of chemical kinetic ordinary differential equations (ODEs) at each computational location (e.g. cell) in the domain~$\rightarrow$~vectorization!
 \end{itemize}

\end{frame}

\end{document}